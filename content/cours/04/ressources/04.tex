\documentclass{cppcourses}

\title[Cours \no 04]{
    \large{Formation C++ 17} \\
    Cours \no 04 \\
    \small{Retour sur la formation C}
}

\begin{document}

\myforeword

\section{Différences avec le C}

\subsection{Surcharge de fonction}

\begin{frame}

\frametitle{Surcharge de fonction}

\begin{definition}
Si le nom d'une fonction réfère à plus d'une entité, alors cette fonction est dite \mydef{surchargée} (ou \mydef{\emph{overloaded}}).
\end{definition}

\begin{remark}<2->
Le compilateur doit alors déterminer quelle \myimp{surcharge} (ou \myimp{\emph{overload}}) appeler. En d'autres termes, la surcharge avec les paramètres correspondant le mieux est appelée.
\end{remark}

\begin{remark}<3->
\myimp{En C}, cette notion n'existe pas : il est impossible de surcharger des fonctions. Ainsi, deux fonctions distinctes doivent avoir nécessairement des noms distincts.
\end{remark}

\end{frame}

\subsubsection{Exemples}

\begin{frame}[fragile]

\frametitle{Un petit exemple}

\begin{example}

\begin{lstlisting}[language = c++]
#include <iostream>

int foo(bool) { return 0; }
int foo(int) { return 1; }

int main()
{
    int a = 42;

    std::cout << foo(a) << '\n';
}
\end{lstlisting}

Dans l'exemple ci-dessus que sera-t-il imprimé en console ?

\myanswer<2->{Il sera imprimé \enquote{1}.}

\end{example}

\end{frame}

\begin{frame}[fragile]

\frametitle{Un petit rappel}

\begin{warning}
On rappelle cependant que le C++ est \myimp{un langage faiblement typé}.
\end{warning}

\begin{example}<2->

\begin{lstlisting}[language = c++]
#include <iostream>

int foo(bool) { return 0; }

int main()
{
    int a = 42;

    std::cout << foo(a) << '\n';
}
\end{lstlisting}

Dans l'exemple ci-dessus que sera-t-il imprimé en console ?

\myanswer<2->{Il sera imprimé \enquote{0}. En effet, un entier peut être implicitement converti en booléen.}

\end{example}

\end{frame}

\subsection{Déclaration implicite de fonction}

\begin{frame}

\frametitle{Déclaration implicite de fonction}

\begin{definition}
Si une fonction n'a pas été déclaré avant son premier appel, celle-ci est \mydef{déclaré implicitement} si le compilateur déduit une signature de cet appel.
\end{definition}

\begin{remark}<2->
Même si cela est à proscrire, \myimp{en C} (du moins, pour les standards antérieurs à \myimp{C 99}), il est possible d'avoir des déclarations implicites.
\end{remark}

\begin{remark}<3->
\myimp{En C++}, il n'existe pas de déclaration implicite.
\end{remark}

\end{frame}

\begin{frame}[fragile]

\frametitle{Un petit exemple}

\begin{example}

\begin{lstlisting}[language = c]
// On n'inclue pas stdio.h

int main()
{
    printf("Hello world!");

    return 0;
}
\end{lstlisting}

Dans l'exemple (écrit en C) ci-dessus, le compilateur pourrait n'imprimer sur la console qu'un avertissement et non se terminer avec une erreur.

\end{example}

\end{frame}

\subsection{Allocation dynamique}

\subsubsection{Allocation dynamique en C}

\begin{frame}

\frametitle{Allocation dynamique en C}

\myimp{En C}, pour \myimp{allouer} ou \myimp{libérer dynamiquement} une ressource, on utilise les fonctions suivantes se trouvant dans l'entête \emph{stdlib.h} :

\begin{myfigure}<2->
\mykeyword{\textcolor{blue}{void}* malloc(\textcolor{blue}{size\_t} size);} \\
\mykeyword{\textcolor{blue}{void}* calloc(\textcolor{blue}{size\_t} num, \textcolor{blue}{size\_t} size);} \\
\mykeyword{\textcolor{blue}{void}* realloc(\textcolor{blue}{void}* ptr, \textcolor{blue}{size\_t} new\_size);} \\
\mykeyword{\textcolor{blue}{void}* aligned\_alloc(\textcolor{blue}{size\_t} alignment, \textcolor{blue}{size\_t} size);}
\caption{Allocation dynamique en C}
\end{myfigure}

\begin{myfigure}<3->
\mykeyword{\textcolor{blue}{void} free(\textcolor{blue}{void}* ptr);}
\caption{Libération dynamique en C}
\end{myfigure}

\end{frame}

\begin{frame}

\frametitle{Une petite observation}

\begin{remark}

Le type de retour des différentes fonctions est, à chaque fois, un \mykeyword{\textcolor{blue}{void}*}. En effet, ce type, comme expliqué dans le cours précédent, correspond à \myimp{un type pointeur générique}.

Grossièrement, comme \myimp{la taille en mémoire caractérise la complétude d'un type} et que \myimp{tous les pointeurs font la même taille}, \mykeyword{\textcolor{blue}{void}*} est bien \myimp{un type complet} même si \mykeyword{\textcolor{blue}{void}} ne l'est pas.

De plus, il existe une conversion implicite de \mykeyword{\textcolor{blue}{type}*} vers \mykeyword{\textcolor{blue}{void}*}.

\end{remark}

\end{frame}

\subsubsection{Allocation dynamique en C++}

\begin{frame}

\frametitle{Allocation dynamique en C++}

\myimp{En C++}, on préfère utiliser les deux opérateurs suivants :

\begin{columns}

\begin{column}{0.5 \textwidth}

\begin{myfigure}<1->
\mykeyword{new} \\
\mykeyword{new[]}
\caption{Allocation dynamique en C++}
\end{myfigure}

\end{column}

\begin{column}{0.5 \textwidth}

\begin{myfigure}<2->
\mykeyword{delete} \\
\mykeyword{delete[]}
\caption{Libération dynamique en C++}
\end{myfigure}

\end{column}

\end{columns}

\end{frame}

\begin{frame}[fragile]

\frametitle{Une petite mise en pratique}

\begin{columns}

\begin{column}{0.45 \textwidth}

\begin{example}

\begin{lstlisting}[language = c++]
#include <iostream>

int main()
{
    int* a = new int(10);

    std::cout << "a = " << *a << '\n';

    delete a;
}
\end{lstlisting}

\end{example}

\end{column}

\begin{column}{0.45 \textwidth}

\begin{example}

\begin{lstlisting}[language = c++]
#include <cstddef>

int main()
{
    int* a = new int[10];

    for (std::size_t i = 0; i != 10; ++i) {
        a[i] = i;
    }

    delete[] a;
}
\end{lstlisting}

\end{example}

\end{column}

\end{columns}

\end{frame}

\begin{frame}

\frametitle{Utilisation avancée de ces opérateurs}

\begin{remark}

Il existe des utilisations plus avancées de ces deux opérateurs que l'on détaillera peut-être plus tard dans la formation.

On peut citer :

\begin{itemize}
    \item<2-> allocation dans \myimp{un tampon} avec \myimp{le \emph{placement new}},
    \item<3-> allocation sans \myimp{exception},
    \item<4-> allocation avec \myimp{contrainte d'alignement},
    \item<5-> \myimp{\emph{handler}}.
\end{itemize}

\end{remark}

Pour les curieux, voir \url{https://en.cppreference.com/w/cpp/memory/new}.

\end{frame}

\subsection{Opérateurs de conversion}

\subsubsection{Les conversions explicites en C++}

\begin{frame}

\frametitle{Les conversions explicites en C++}

\myimp{Le C++} est doté de plusieurs opérateurs de conversion différents ayant chacun un objectif propre.

\begin{myfigure}
\mykeyword{const\_cast<\textcolor{blue}{<type cible>}>(\textcolor{orange}{<expression>})} \\
\mykeyword{static\_cast<\textcolor{blue}{<type cible>}>(\textcolor{orange}{<expression>})} \\
\mykeyword{dynamic\_cast<\textcolor{blue}{<type cible>}>(\textcolor{orange}{<expression>})} \\
\mykeyword{reinterpret\_cast<\textcolor{blue}{<type cible>}>(\textcolor{orange}{<expression>})}
\caption{Différents opérateurs de conversions en C++}
\end{myfigure}

\end{frame}

\begin{frame}[fragile]

\frametitle{\mykeyword{const\_cast}}

\begin{myfigure}
\mykeyword{const\_cast<\textcolor{blue}{<type cible>}>(\textcolor{orange}{<expression>})} \\
\caption{Utilisation de \mykeyword{const\_cast}}
\end{myfigure}

Il permet la conversion de \mykeyword{\textcolor{orange}{expression}} en un type \mykeyword{\textcolor{blue}{type cible}} de constance différente.

\begin{example}<2->

\begin{lstlisting}[language = c++]
#include <iostream>

int main()
{
    const int* p_a = new int(5);

    std::cout << *p_a << '\n';

    *const_cast<int*>(p_a) = 10;

    std::cout << *p_a << '\n';
}
\end{lstlisting}

Dans l'exemple ci-dessus que sera-t-il imprimer sur la ligne de commande ?

\myanswer<3->{Il sera imprimé \enquote{5}, puis \enquote{10}.}

\end{example}

\end{frame}

\begin{frame}[fragile]

\frametitle{\mykeyword{static\_cast}}

\begin{myfigure}
\mykeyword{static\_cast<\textcolor{blue}{<type cible>}>(\textcolor{orange}{<expression>})} \\
\caption{Utilisation de \mykeyword{static\_cast}}
\end{myfigure}

Il permet la conversion de \mykeyword{\textcolor{orange}{expression}} en un type \mykeyword{\textcolor{blue}{type cible}} en utilisant une combinaison de règles de conversion implicites et d'autres fournies par l'utilisateur.

\begin{example}<2->

\begin{lstlisting}[language = c++]
#include <iostream>

int main()
{
    double a = 1.5;

    std::cout << static_cast<int>(a) << '\n';
}
\end{lstlisting}

Dans l'exemple ci-dessus que sera-t-il imprimer sur la ligne de commande ?

\myanswer<3->{Il sera imprimé \enquote{1}.}

\end{example}

\end{frame}

\begin{frame}[fragile]

\frametitle{\mykeyword{reinterpret\_cast}}

\begin{myfigure}
\mykeyword{reinterpret\_cast<\textcolor{blue}{<type cible>}>(\textcolor{orange}{<expression>})} \\
\caption{Utilisation de \mykeyword{reinterpret\_cast}}
\end{myfigure}

Il permet la conversion de \mykeyword{\textcolor{orange}{expression}} en un type \mykeyword{\textcolor{blue}{type cible}} en réinterprétant la représentation binaire sous-jacente.

\begin{example}<2->

\begin{lstlisting}[language = c++]
#include <cstdint>
#include <iostream>

int main()
{
    std::uint32_t a = 0x00544942;

    std::cout << reinterpret_cast<char*>(&a) << '\n';
}
\end{lstlisting}

Dans l'exemple ci-dessus que sera-t-il imprimer sur la ligne de commande ?

\myanswer<3->{En supposant que l'on travaille sur une machine petit-boutiste, il sera imprimé \enquote{BIT}.}

\end{example}

\end{frame}

\subsubsection{Les conversions explicites en C}

\begin{frame}

\frametitle{Les conversions explicites en C}

\begin{myfigure}
\mykeyword{(\textcolor{blue}{<type cible>}) \textcolor{orange}{<expression>}} \\
\mykeyword{\textcolor{blue}{<type cible>} (\textcolor{orange}{<expression>})}
\caption{Conversion explicite en C}
\end{myfigure}

\uncover<2->{Le comportement est équivalent à une tentative de conversion dans l'ordre suivant :}

\begin{enumerate}
    \item<3-> \mykeyword{const\_cast}
    \item<4-> \mykeyword{static\_cast}
    \item<5-> \mykeyword{dynamic\_cast}
    \item<6-> \mykeyword{reinterpret\_cast}
\end{enumerate}

\end{frame}

\subsection{Mots clefs}

\begin{frame}

\frametitle{Mots clefs}

Bien que \myimp{le C++} descends du \myimp{C} et que nombreux mots-clefs du \myimp{C} se retrouve également en \myimp{C++}, \myimp{le C} a suivi sa propre évolution : par conséquent, certains mots clefs existent en \myimp{C} et non en \myimp{C++}.

\begin{remark}<2->
Cependant, certains des mots clefs qui ont été introduit en \myimp{C} l'ont également été en \myimp{C++} parallèlement même s'\myimp{ils ne donc suivent pas nécessairement la même syntaxe}.
\end{remark}

\end{frame}

\subsubsection{\mykeyword{restrict}}

\begin{frame}

\frametitle{\mykeyword{restrict}}

\begin{myfigure}
\mykeyword{\textcolor{blue}{<type>}* restrict}
\caption{Utilisation de \mykeyword{restrict}}
\end{myfigure}

\begin{uncoverenv}<2->

Marquer un pointeur comme \mykeyword{restrict} indique au compilateur que l'entité visée par ce pointeur ne peut être \myimp{lu} ou \myimp{modifié}, \myimp{directement} ou \myimp{indirectement} que par ledit pointeur.

Ce mot clef permet au compilateur certaines optimisations, en plus d'indiquer au lecteur le comportement de l'entité spécifiée.

\end{uncoverenv}

\begin{warning}<3->
Si les contraintes définies ci-dessus ne sont pas respectées, alors le comportement est \myimp{indéfini}.
\end{warning}

\begin{remark}<4->
\mykeyword{restrict} caractérise le type : c'est-à-dire que \mykeyword{\textcolor{blue}{type}* restrict} et \mykeyword{\textcolor{blue}{type}*} sont deux types distincts.
\end{remark}

\end{frame}

\begin{frame}[fragile]

\frametitle{Une petite mise en contexte}

\begin{example}

\begin{lstlisting}[language = c]
void copy(int n, int* restrict p, int* restrict q)
{
    while (n-- > 0) {
        *p++ = *q++;
    }
}

int main()
{
    int d[100];

    for (size_t i = 0; i != 50; ++i) {
        d[i] = i;
    }

    copy(50, d + 50, d);
    copy(50, d + 1, d);
}
\end{lstlisting}

Dans l'exemple ci-dessus que va-t-il se passer lors des deux appels à \mykeyword{copy()} ?

\myanswer<2->{Le premier valide car \mykeyword{p} et \mykeyword{q} ne référeront jamais au même élément ; cependant, ce n'est pas le cas du second appel.}

\end{example}

\end{frame}

\subsubsection{D'autres mots clefs}

\begin{frame}

\frametitle{D'autres mots clefs}

\begin{columns}<only@-4>

\begin{column}{0.5 \textwidth}

\begin{myfigure}
\mykeyword{\_Alignas} \\
\mykeyword{\_Alignof}
\caption{Mots clefs relatifs à l'alignement}
\end{myfigure}

\begin{myfigure}<2->
\mykeyword{\_Atomic}
\caption{Spécifieur pour l'atomicité}
\end{myfigure}

\end{column}

\begin{column}{0.5 \textwidth}

\begin{myfigure}<3->
\mykeyword{\_Generic}
\caption{Mot clef pour la sélection générique}
\end{myfigure}

\begin{myfigure}<4->
\mykeyword{\_Bool}
\caption{Type booléen}
\end{myfigure}

\end{column}

\end{columns}

\begin{columns}<only@5->

\begin{column}{0.5 \textwidth}

\begin{myfigure}<5->
\mykeyword{\_Complex} \\
\mykeyword{\_Imaginary}
\caption{Mots clefs relatifs aux nombres complexes}
\end{myfigure}

\begin{myfigure}<6->
\mykeyword{\_Thread\_local}
\caption{Spécifieur de durée de stockage}
\end{myfigure}

\end{column}

\begin{column}{0.5 \textwidth}

\begin{myfigure}<7->
\mykeyword{\_Static\_assert}
\caption{Mot clef pour les assertions statiques}
\end{myfigure}

\end{column}

\end{columns}

\end{frame}

\section{Notions manquantes}

\subsection{Notion de structure}

\begin{frame}

\frametitle{Notion de structure}

\begin{myfigure}
\mykeyword{struct \textcolor{red}{[identifier]};}
\caption{Déclaration d'une structure}
\end{myfigure}

\begin{myfigure}
\mykeyword{struct \textcolor{red}{[identifier]} \{ \textcolor{blue}{[corps]} \};}
\caption{Définition d'une structure}
\end{myfigure}

\begin{definition}<2->
\mydef{Une structure} est un type consistant en \myimp{un agrégat} de données.
\end{definition}

\end{frame}

\begin{frame}[fragile]

\frametitle{Un petit exemple}

\begin{example}

\begin{lstlisting}[language = c++]
#include <cstdint>
#include <iostream>

struct Player { char* name; std::uint8_t exp; };

int main()
{
    Player matteo = { "Mattéo", 255 }; // initialisation aggrégat

    std::cout << matteo.name << '\n';
    std::cout << matteo.exp << '\n';
}
\end{lstlisting}

Que sera-t-il imprimé sur la console ?

\myanswer<2->{Il sera imprimé \enquote{Mattéo}, puis \enquote{255}.}

\end{example}

\end{frame}

\subsection{L'union fait la force}

\begin{frame}

\frametitle{L'union fait la force}

\begin{myfigure}
\mykeyword{union \textcolor{red}{[identifier]} \{ \textcolor{blue}{[corps]} \};}
\caption{Déclaration d'un union}
\end{myfigure}

\begin{definition}<2->
\mydef{Un union} est un type spécial de structure où seul un membre vit à la fois.
\end{definition}

\end{frame}

\begin{frame}[fragile]

\frametitle{Un petit exemple}

\begin{example}

\begin{lstlisting}[language = c++]
#include <cstdint>
#include <iostream>

union Union { std::uint32_t integer; char character; };

int main()
{
    Union u = {0x00544942};

    std::cout << &u.character << '\n';
}
\end{lstlisting}

Dans l'exemple ci-dessus que sera-t-il imprimé sur la ligne de commande ?

\myanswer<2->{Il sera imprimé \enquote{BIT}.}

\end{example}

\end{frame}

\subsection{Nommer des constantes avec les énumérations}

\begin{frame}

\frametitle{Nommer des constantes avec les énumérations}

\begin{myfigure}
\mykeyword{<enum> \textcolor{red}{[identifier]} : \textcolor{orange}{<base>};}
\caption{Déclaration d'une énumération}
\end{myfigure}

\begin{myfigure}
\mykeyword{<enum> \textcolor{red}{[identifier]} \{ \textcolor{blue}{[corps]} \};} \\
\mykeyword{<enum> \textcolor{red}{[identifier]} : \textcolor{orange}{<base>} \{ \textcolor{blue}{[corps]} \};}
\caption{Définition d'une énumération}
\end{myfigure}

\begin{itemize}
    \item<2-> \mykeyword{<enum>} peut prendre les valeurs \mykeyword{enum}, \mykeyword{enum class} ou \mykeyword{enum struct}.
    \item<3-> \myimp{Le type entier} des valeurs définies est \mykeyword{\textcolor{orange}{<base>}}.
\end{itemize}

\begin{definition}<4->
\mydef{Une énumération} est un type distincts définissant une liste de valeurs nommées dans une plage donnée.
\end{definition}

\end{frame}

\begin{frame}[fragile]

\frametitle{Un petit exemple}

\begin{example}

\begin{lstlisting}[language = c++]
#include <cstdint>
#include <iostream>

enum Enum : uint8_t { first, second = 10, third };
enum class EnumClass { first, second = 10, third };

int main()
{
    int a = first;
    Enum b = 10;
    EnumClass c = EnumClass::first;

    std::cout << a << ", " << b << ", " << c << '\n';
}
\end{lstlisting}

Que va-t-il se passer et pourquoi ?

\end{example}

\end{frame}

\mybye

\end{document}
