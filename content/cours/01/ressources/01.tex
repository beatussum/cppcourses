\documentclass{cppcourses}

\title[Formation C++ 17]{
    Formation C++ 17 \no 01 \\
    \small{Variables, types et notion de programmation impérative}
}

\begin{document}

\frame{\titlepage}
\frame{\tableofcontents[pausesections, pausesubsections]}

\section{Avant-propos}

\begin{frame}

\frametitle{Avant-propos}

Avant de commencer, il est important de rappeler que ce cours est réalisé par \myimp{un étudiant}.

Par conséquent, il n'a pas la même fiabilité qu'un cours dispensé par \myimp{un réel enseignant de l'ENSIMAG}.

N'utilisez pas ce cours comme \myimp{un argument d'autorité} !

Si un professeur semble, a posteriori, contredire des éléments apportés par ce cours, \myimp{il a très probablement raison}.

Ce document est \myimp{vivant} : je veillerai à corriger les coquilles ou erreurs plus problématiques.

\end{frame}

\section{Notion de programmation impérative}

\subsection{Introduction}

\begin{frame}

\frametitle{Notion de programmation impérative}

\begin{definition}
\mydef{La programmation impérative} est \myimp{un paradigme de programmation}, une façon d'approcher la programmation, qui considère les opérations comme des séquences d'instructions exécutées par l'ordinateur pour modifier l'état du programme.
\end{definition}

De manière plus schématique, cela revient (presque) à lire le programme dans l'ordre de lecture usuel.

\end{frame}

\begin{frame}

\frametitle{Notion de \emph{statements}}

\begin{definition}
\mydef{Des \emph{statements}} sont des fragments d'un programme C++ exécutés en séquence.
\end{definition}

En C++, tout \emph{statement} se termine par \myimp{un point-virgule} (\myimp{\mykeyword{;}}).

\mywarn<2->{Le terme \enquote{\emph{statement}} pourrait être traduit par \enquote{instruction} en français ; cependant, ce terme est ambigu car il pourrait également référer à \myimp{des instructions machines} : on préférera donc, dans ce cours, parler de \emph{statement}.}

\end{frame}

\begin{frame}[fragile]

\frametitle{Commentaires}

\begin{definition}
\mydef{Un commentaire} est un texte documentaire qui n'a aucun impact sur le programme : son contenu est ignoré par le compilateur.
\end{definition}

En C++, la syntaxe utilisée est la suivante :

\begin{lstlisting}[language = c++]
// Un commentaire sur une ligne.

/*
 * Un commentaire sur
 * une ou plusieurs lignes.
 */
\end{lstlisting}

\end{frame}

\begin{frame}[fragile]

\frametitle{Imprimer sur la console}

Pour afficher du texte sur la console, on mettra au début de tous les fichiers les lignes suivantes :

\begin{lstlisting}[language = c++]
#include <iostream>

using namespace std;
\end{lstlisting}

Pour afficher \enquote{txt}, on utilisera :

\begin{lstlisting}[language = c++]
cout << txt;
\end{lstlisting}

Si on veut rajouter un retour à la ligne, on préférera :

\begin{lstlisting}[language = c++]
cout << txt << '\n';
\end{lstlisting}

\end{frame}

\subsection{La fonction \mykeyword{main}}

\begin{frame}[fragile]

\frametitle{La fonction \mykeyword{main}}

En C++, comme en C, le point de départ de l'exécution du programme est la première instruction de la fonction \mykeyword{main}.

\begin{lstlisting}[language = c++]
#include <iostream>

using namespace std;

int main(int argc, char* argv[])
{
    cout << "Hello world!" << '\n';

    return 0; // cette ligne est optionnelle
}
\end{lstlisting}

\mywarn{Particularité de cette fonction (et uniquement celle-ci), on peut omettre le \mykeyword{return 0} car c'est la valeur de retour par défaut.}

\end{frame}

\begin{frame}[fragile]

\frametitle{La fonction \mykeyword{main}}

\begin{lstlisting}[language = c++]
int main(int argc, char* argv[])
\end{lstlisting}

Cette fonction a pour type de retour un \mykeyword{int} (un entier). Celui-ci notifie si le programme s'est terminé normalement :

\begin{itemize}
    \item<2-> un retour nul signifie qu'il n'y a pas eu d'erreur,
    \item<3-> à l'inverse d'une valeur non nul qui informe, malgré cela, l'erreur rencontrée par sa valeur.
\end{itemize}

\uncover<4->{Elle prend deux arguments : \myimp{un entier} et \myimp{un tableau de chaînes de caractères}. Par convention, on a :}

\begin{itemize}
    \item<5-> \mykeyword{argc} (\emph{argument count}) qui représente le nombre d'argument passé en ligne de commande,
    \item<6-> \mykeyword{argv} (\emph{argument vector}) qui représente les arguments.
\end{itemize}

\mywarn<7->{Au moins un argument est passé au programme : il s'agit de son nom (son chemin).}

\end{frame}

\begin{frame}[fragile]

\frametitle{Différentes formes possibles}

\begin{columns}
    \begin{column}{0.5\textwidth}
\footnotesize

\mykeyword{argv} est ici \myimp{un tableau de chaîne de caractères}.

        \begin{lstlisting}[language = c++]
#include <iostream>

using namespace std;

int main(int argc, char* argv[])
{
    cout << "Hello world!" << '\n';
}
        \end{lstlisting}

\mykeyword{argv} est ici \myimp{un pointeur vers le premier élément d'un tableau de chaînes de caractères}.

        \begin{lstlisting}[language = c++]
#include <iostream>

using namespace std;

int main(int argc, char** argv)
{
    cout << "Hello world!" << '\n';
}
        \end{lstlisting}
    \end{column}
    \begin{column}{0.5\textwidth}
\footnotesize

Quand on a besoin ni de \mykeyword{argc} ni de \mykeyword{argv}, on peut les omettre.

        \begin{lstlisting}[language = c++]
#include <iostream>

using namespace std;

int main()
{
    cout << "Hello world!" << '\n';
}
        \end{lstlisting}
    \end{column}
\end{columns}

\end{frame}

\subsubsection{Quelques exemples}

\begin{frame}[fragile]

\frametitle{Pour bien comprendre la notion de point d'entrée}

\begin{example}
    \begin{lstlisting}[language = c++]
#include <iostream>

using namespace std;

void foo()
{
    cout << "foo" << '\n';
}

void hello()
{
    cout << "Hello world!" << '\n';
}

int main(int argc, char* argv[])
{
    hello();
}
     \end{lstlisting}

Dans l'exemple ci-dessus, que sera-t-il affiché sur la console ?

\myanswer<2->{Il sera affiché \enquote{Hello world!}.}
\end{example}

\end{frame}

\begin{frame}[fragile]

\frametitle{Pour bien comprendre la notion de point d'entrée}

\begin{example}
    \begin{lstlisting}[language = c++]
#include <iostream>

using namespace std;

void hello()
{
    cout << "Hello world!" << '\n';
}

hello();

int main(int argc, char* argv[])
{
    hello();
}
     \end{lstlisting}

Dans l'exemple ci-dessus, que va-t-il se passer et pourquoi ?

\myanswer<2->{Le programme ne se compilera pas car on appelle une fonction dans la portée globale.}
\end{example}

\end{frame}

\section{Définition de variables}

\subsection{Introduction}

\begin{frame}

\frametitle{Un langage à typage statique et faible}

\begin{definition}
\mydef{Un langage statiquement typé} est un langage dont le type des variables est déterminé à la compilation. Il s'oppose à \myimp{un langage dynamiquement typé}.
\end{definition}

\begin{definition}
\mydef{Un langage faiblement typé} est un langage dont chaque variable, bien qu'elle possède un type, peut en changer en s'appuyant, par exemple, sur des règles de conversion.
\end{definition}

Le C++ est \myimp{un langage à typage statique et faible} ; à l'inverse, Python est \myimp{un langage à typage dynamique et fort}.

\end{frame}

\begin{frame}[fragile]

\frametitle{L'intérêt d'un typage statique}

\begin{example}

\begin{onlyenv}<1>

Considérons le programme Python ci-dessous :

\begin{lstlisting}[language = Python]
#! /usr/bin/env python3


def my_sum(a, b):
    """
    Somme deux entiers.
    """

    return a + b

def main():
    my_sum(10, 10)
    my_sum([1], [3]) # fonctionne quand même !
    my_sum(10, "toto") # on a une erreur ici !

if __name__ == "__main__":
    main()
\end{lstlisting}

\end{onlyenv}

\only<2->{On remarque deux choses :}

\begin{enumerate}
    \item<only@2- | uncover@3-> à \textbf{la ligne 13}, on a aucune erreur bien que l'on ne veuille que des entiers ;
    \item<only@2- | uncover@4-> à \textbf{la ligne 14}, on a une erreur qui ne se manifeste que lorsque la ligne est atteinte à l'exécution.
\end{enumerate}

\action<only@2- | uncover@5->{En C++, les variables étant statiquement typées, les deux problématiques ci-dessus ne se posent pas.}

\end{example}

\end{frame}

\begin{frame}

\frametitle{Notion d'alias}

En C++, comme en C, on peut nommer un \mykeyword{\textcolor{red}{type}} par un \mykeyword{\textcolor{blue}{autre type}}.

\uncover<2->{Il existe deux syntaxes équivalentes :}

\begin{uncoverenv}<3->

\begin{figure}
\mykeyword{typedef \textcolor{red}{<type>} \textcolor{blue}{<autre nom>};}
\caption{définition d'un alias (syntaxe C)}
\end{figure}

\end{uncoverenv}

\begin{uncoverenv}<4->

\begin{figure}
\mykeyword{using \textcolor{blue}{<autre nom>} = \textcolor{red}{<type>};}
\caption{définition d'un alias (syntaxe C++)}
\end{figure}

\end{uncoverenv}

\end{frame}

\begin{frame}[fragile]

\frametitle{Un exemple}

\begin{example}

\begin{lstlisting}[language = c++]
#include <iostream>

using namespace std;

using mon_super_booleen = int;
typedef int mon_super_type;

int main()
{
    mon_super_entier a = 5;
    mon_super_type b   = 10;

    cout << a << '\n';
    cout << b << '\n';
}
\end{lstlisting}

Dans l'exemple ci-dessus que va-t-il s'imprimer sur la console ?

\myanswer<2->{Il y aura \enquote{5}, puis \enquote{10}.}

\end{example}

\end{frame}

\subsection{Types fondamentaux}

\begin{frame}

\frametitle{Types fondamentaux}

Le C++ est \myimp{un langage fortement typé} : toutes les variables ont un type et qui ne peut pas changer au cours de l'exécution du programme.

Pour l'instant, on se limitera au \myimp{types fondamentaux} (ou \myimp{fundamental types}) à l'exception de \mykeyword{std::nullptr\_t} que l'on verra plus tard quand on parlera de pointeur.

\uncover<2->{Il s'agit de :}

\begin{itemize}
     \item<2-> \mykeyword{void} (presque équivalent au \mykeyword{None} de Python),
     \item<3-> \mykeyword{bool} (\myimp{booléen}),
     \item<4-> \mykeyword{char}, \mykeyword{signed char} et \mykeyword{unsigned char} (\myimp{types caractères ordinaires}),
     \item<5-> \mykeyword{char16\_t}, \mykeyword{char32\_t} et \mykeyword{wchar\_t} (\myimp{\emph{wide character types}}),
     \item<6-> \mykeyword{float}, \mykeyword{double} et \mykeyword{long double} (\myimp{types virgules flottantes}),
     \item<7-> \mykeyword{signed char}, \mykeyword{short}, \mykeyword{int}, \mykeyword{long} et \mykeyword{long long} (\myimp{types entiers signés}),
     \item<8-> \mykeyword{unsigned char}, \mykeyword{unsigned short}, \mykeyword{unsigned int}, \mykeyword{unsigned long} et \mykeyword{unsigned long long} (\myimp{types entiers non signés}).
\end{itemize}

\end{frame}

\subsubsection{Les types non entiers}

\begin{frame}[fragile]

\frametitle{Le mot clef \mykeyword{void}}

\begin{definition}
\mydef{Un type incomplet} est un type qui manque des informations nécessaires pour connaître sa taille en mémoire.
\end{definition}

Une variable de type incomplet est très limitée dans son utilisation.

\mykeyword{void} est un type incomplet qui ne peut pas être complété. Son rôle sera vu, plus en détails, quand on parlera de pointeurs, de fonctions et de \mykeyword{template}.

\begin{example}<2->

De cette façon, la ligne ci-dessous ne compilera pas :

\begin{lstlisting}[language = c++]
void a;
\end{lstlisting}

\end{example}

\end{frame}

\begin{frame}[fragile]

\frametitle{Les booléens}

Un booléen a deux valeurs possibles :

\begin{itemize}
    \item<2-> \mykeyword{false},
    \item<3-> \mykeyword{true}.
\end{itemize}

\uncover<4->{N'importe quel intégral (\emph{integral}) (un caractère ou un entier) ou flottant peut être converti implicitement en booléen. La règle est la suivante :}

\begin{itemize}
    \item<5-> si sa valeur est nulle, alors la variable en question vaut \mykeyword{false},
    \item<6-> sinon, il vaut \mykeyword{true}.
\end{itemize}

\uncover<7->{Ceci constitue un des nombreux exemples montrant que le C++ est \myimp{un langage faiblement typé}.}

\begin{example}<8->
    \begin{lstlisting}[language = c++]
bool a = 255; // a est vrai
bool b = 0.; // b est faux
bool c = '0'; // b est vrai
    \end{lstlisting}
\end{example}

\end{frame}

\begin{frame}

\frametitle{Types caractères ordinaires}

\mykeyword{char} est soit \myimp{signé} soit \myimp{non signé} mais il diffèrera toujours de
\mykeyword{signed char} et \mykeyword{unsigned char}.

\mykeyword{char} est capable de stocker un certain nombre de caractères Unicode dont on peut trouver la liste précise sur \url{https://en.cppreference.com/w/cpp/language/charset\#Basic\_character\_set}.

Il a une taille fixe : dans presque toutes les situations, elle est d'\myimp{un octet}.

\end{frame}

\begin{frame}

\frametitle{\emph{Wide character types}}

\begin{definition}
La taille des caractères dans certains encodages (par exemple, \myimp{UTF-8}, \myimp{UTF-16} et \myimp{UTF-32}) est variables. De cette façon, un caractère est divisé en plusieurs unités appelées \mydef{unités de code}.
\end{definition}

Les types étudiés ici représentent chacun \myimp{une unité de code}.

\uncover<2->{On retrouve :}

\begin{itemize}
    \item<2-> \mykeyword{char16\_t} en \myimp{UTF-16} (\myimp{au moins deux octets}),
    \item<3-> \mykeyword{char32\_t} en \myimp{UTF-32} (\myimp{au moins quatre octets}),
    \item<4-> \mykeyword{wchar\_t} qui est sensé supporté n'importe quelle unité de code qu'importe l'encodage utilisé.
\end{itemize}

\mywarn<5->{En réalité, \mykeyword{wchar\_t} est dysfonctionnel et son usage est à limiter.}

\end{frame}

\begin{frame}

\frametitle{Types virgules flottantes}

Il existe trois types différents :

\begin{enumerate}
    \item \mykeyword{float} est \myimp{une virgule flottante simple précision},
    \item<2-> \mykeyword{double} est \myimp{une virgule flottante double précision},
    \item<3-> \mykeyword{long double} est, si la machine cible la supporte, \myimp{une virgule flottante quadruple précision} ; sinon si cela est supporté, il s'agit d'\myimp{une virgule flottante double précision étendue} ; sinon si celui-ci existe, un autre format avec une meilleure précision ; sinon, c'est \myimp{une virgule flottante double précision}.
\end{enumerate}

\mywarn<4->{En d'autres termes, les types ci-dessus sont listés par ordre de précision croissante (pas strictement).}

\end{frame}

\subsubsection{Les entiers}

\begin{frame}

\frametitle{Une histoire de signe}

Un entier est dit \enquote{\myimp{signé}} s'il est capable de supporter des valeurs négatives ; dans le cas contraire, on dit qu'il est \enquote{\myimp{non signé}}.

\begin{uncoverenv}<2->

Pour définir un entier comme signé, on peut utiliser la syntaxe suivante :

\begin{figure}
\mykeyword{unsigned \textcolor{red}{<type entier>}}
\caption{forme d'un entier non signé}
\end{figure}

\end{uncoverenv}

\uncover<3->{Par défaut, en C++, les entiers sont signés ; on peut cependant rajouter le mot clef \myimp{\mykeyword{signed}} pour l'expliciter.}

\end{frame}

\begin{frame}

\frametitle{Comment représenter un entier négatif en mémoire ?}

On peut proposer trois approches différentes :

\begin{enumerate}

    \item<2-> l'utilisation d'un bit de signe,

    \myissue<only@3-4>{
        \begin{align*}
            \begin{split}
{0001}_2 + {1001}_2 &= {1010}_2 \\
\myhence 1 + (-1) &= -2
            \end{split}
        \end{align*}

On devrait donc avoir une interprétation pas très pratique des calculs arithmétiques avec les entiers signés.
    }

    \item<4-> l'opposé d'un entier est son complément à 1 (bit-à-bit),

    \begin{onlyenv}<5-7>

    On a dorénavant bien :

    \begin{align*}
        \begin{split}
    {0001}_2 + {1110}_2 &= {1111}_2 \\
    \myhence 1 + (-1) &= 0
        \end{split}
    \end{align*}

    \end{onlyenv}

    \myissue<6-7 | only@5-7>{On a deux représentation de 0 qui sont \( {0000}_2 \) et \( {1111}_2 \).}

     \item<7- | alert@9-> l'opposé d'un entier sur \( n \) bits est son complément à \( 2^n \) (abusivement appelé son complément à 2).

\begin{onlyenv}<8>
La représentation sur quatre bits de l'opposé de 1 est donc \( 2^4 - 1 = 15 = {1111}_2 \).

On a donc :

\begin{align*}
    \begin{split}
{0001}_2 + {1111}_2 &= {\cancel{1}0000}_2 = {0000}_2 \\
\myhence 1 + (-1) &= 0
    \end{split}
\end{align*}

\( {0000}_2 \) est aussi la seule représentation de 0.
\end{onlyenv}

\end{enumerate}

\begin{onlyenv}<9->
Pour représenter \myimp{un entier signé}, un entier qui peut avoir des valeurs négatives, les concepteurs d'architecture ont tous choisi la troisième méthode.

Un entier signé sur \( n \) bits prend donc ses valeurs de \( -2^{n - 1} \) à \( 2^{n - 1} - 1\).

Le bit de poids fort caractérise le signe de l'entier.

On a, pour un nombre \( x \) encodé sur \( n \) bits :

\begin{align*}
    \begin{split}
2^n - x &= \lnot x + 1 \\
\text{ou encore } (1 \ll n) - x &= \lnot x + 1
    \end{split}
\end{align*}
\end{onlyenv}

\end{frame}

\begin{frame}

\frametitle{Les entiers et leur taille en mémoire}

En C++, le standard est trompeur : la taille spécifiée des entiers n'est pas stricte mais minimale.

\uncover<2->{Ainsi, on a :}

\begin{enumerate}
    \item<2-> \mykeyword{char} au moins \myimp{un octet},
    \item<3-> \mykeyword{short} au moins \myimp{deux octets},
    \item<4-> \mykeyword{int} au moins \myimp{deux octets},
    \item<5-> \mykeyword{long} au moins \myimp{trois octets},
    \item<6-> \mykeyword{long long} au moins \myimp{quatre octets}.
\end{enumerate}

\end{frame}

\begin{frame}

\frametitle{Les modèles de données}

La taille exacte de ces types dépend en réalité du \myimp{modèle de données} utilisé et dépend donc de :

\begin{itemize}
    \item<2-> la taille des \myimp{mots} de la machine cible,
    \item<3-> le système d'exploitation de cette même machine.
\end{itemize}

\end{frame}

\begin{frame}

\frametitle{Des alias bien pratiques}

Comme les mots clefs précédents rendent la portabilité du code très complexe, on utilise, dans les faits, des alias qui assure que la taille de l'entier soit la bonne :

\begin{center}
    \begin{tabular}{||c||c||c|c||}
\multirow{2}{*}{Type} & \multirow{2}{*}{Taille en mémoire (en octets)} & \multicolumn{2}{c||}{Plage de valeurs} \\
\cline{3-4}
&& Entier non signé & Entier signé \\
\hline \hline
\mykeyword{int8\_t} & 1 & \numrange{0}{255} & \numrange{-128}{127} \\
\mykeyword{int16\_t} & 2 & \numrange{0}{65535} & \numrange{-32768}{32767} \\
\mykeyword{int32\_t} & 4 & \numrange{0}{4,29e9} & \num{\pm 2,14e9} \\
\mykeyword{int64\_t} & 8 & \numrange{0}{1,84e19} & \num{\pm 9,22e18}
    \end{tabular}
\end{center}

\end{frame}

\begin{frame}[fragile]

\frametitle{Dernières précisions}

La version non signée de \mykeyword{int\( n \)\_t} est \myimp{\mykeyword{uint\( n \)\_t}}.

\myimp{\mykeyword{int\_least\( n \)\_t}} assure que l'entier fait au moins \( n \) bits.

\myimp{\mykeyword{int\_fast\( n \)\_t}} assure que l'entier est le plus rapide et fait au moins \( n \) bits.

Pour utiliser ces alias, il faut mettre en début de fichier les lignes :

\begin{lstlisting}[language = c++]
#include <cstdint>

using namespace std; // ne pas mettre plusieurs fois cette ligne
\end{lstlisting}

\end{frame}

\subsection{Les littéraux}

\begin{frame}[fragile]

\frametitle{Les littéraux}

\begin{definition}
\mydef{Les littéraux} (ou \mydef{litterals}) sont, en C++, des unités syntaxiques qui représente les valeurs constantes intégrées aux langage.
\end{definition}

\begin{example}<2->

Dans l'exemple ci-dessous, les unités syntaxiques à droites des signes \myimp{\mykeyword{=}} sont des littéraux.

\begin{lstlisting}[language = c++]
int a = 5;
double b = 5.;
\end{lstlisting}

\end{example}

\end{frame}

\subsubsection{Les littéraux entiers}

\begin{frame}[fragile]

\frametitle{Un entier en décimal}

En décimal, les chiffres autorisés sont \myimp{les chiffres décimaux classiques} :

\begin{figure}
\mykeyword{\textcolor{red}{<chiffres>}\textcolor{orange}{[suffixe]}}
\caption{entier exprimé en décimal}
\end{figure}

\begin{example}<2->

\begin{lstlisting}[language = c++]
int a = 5;
\end{lstlisting}

\end{example}

\end{frame}

\begin{frame}

\frametitle{Une petite précision syntaxique}

\mykeyword{\textcolor{orange}{suffixe}} détermine \myimp{le type de l'entier}.

\begin{itemize}
    \item<2-> \textcolor{orange}{aucun suffixe} : il s'agit au moins d'un \mykeyword{int} ;
    \item<3-> \mykeyword{\textcolor{orange}{l}} ou \mykeyword{\textcolor{orange}{L}} : il s'agit au moins d'un \mykeyword{long int} ;
    \item<4-> \mykeyword{\textcolor{orange}{ll}} ou \mykeyword{\textcolor{orange}{LL}} : il s'agit au moins d'un \mykeyword{long long int}.
\end{itemize}

\uncover<5->{\myimp{En base décimale}, les types sont des \myimp{types signés} ; à l'inverse, \myimp{pour les autres bases}, il peut s'agir de \myimp{types signés} ou \myimp{non signés}.}

\uncover<6->{Pour imposer \myimp{un type non signé}, on peut ajouter (\myimp{potentiellement, en plus de ceux présentés ci-dessus}) le suffixe \mykeyword{\textcolor{orange}{u}} ou \mykeyword{\textcolor{orange}{U}}.}

\end{frame}

\begin{frame}[fragile]

\frametitle{Un entier en octal}

En octal, les chiffres autorisés sont \myimp{les chiffres de \numrange{0}{7}}.

\begin{figure}
\mykeyword{0\textcolor{red}{<chiffres>}\textcolor{orange}{[suffixe]}}
\caption{entier exprimé en octal}
\end{figure}

\mywarn<2->{L'octal n'est plus une base courante.}

\begin{example}<3->

\begin{lstlisting}[language = c++]
unsigned int a = 010u; // a vaut 8 en décimal
\end{lstlisting}

\end{example}

\end{frame}

\begin{frame}[fragile]

En hexadécimal, les chiffres autorisés sont \myimp{les chiffres de \numrange{0}{9} et de a à f} (la syntaxe n'est pas sensible à la casse).

\begin{figure}
\mykeyword{0\textbf{(x|X)}\textcolor{red}{<chiffres>}\textcolor{orange}{[suffixe]}}
\caption{entier exprimé en hexadécimal}
\end{figure}

\begin{example}<2->

\begin{lstlisting}[language = c++]
long long a = 0xaAbCdEfLL; // a vaut 179031535 en décimal
unsigned long long b = 0X10ull; // a vaut 16 en décimal
\end{lstlisting}

\end{example}

\end{frame}

\begin{frame}[fragile]

\frametitle{Les entiers en binaire}

En binaire, les chiffres autorisés sont \myimp{les chiffres 0 et 1}.

\begin{figure}
\mykeyword{0\textbf{(b|B)}\textcolor{red}{<chiffres>}\textcolor{orange}{[suffixe]}}
\caption{entier exprimé en binaire}
\end{figure}

\begin{example}<2->

\begin{lstlisting}[language = c++]
unsigned int a = 0B10u; // a vaut 2 en décimal
int b = 0b01; // a vaut 1 en décimal
\end{lstlisting}

\end{example}

\end{frame}

\subsubsection{Les littéraux virgules flottantes}

\begin{frame}

\frametitle{Les flottants en décimal}

\begin{figure}
\mykeyword{\textcolor{red}{<chiffres>}\textbf{(e|E)}\textcolor{blue}{<exposant>}\textcolor{orange}{[suffixe]}} \\
\mykeyword{.\textcolor{red}{<chiffres>}\textbf{(e|E)}\textcolor{blue}{[exposant]}\textcolor{orange}{[suffixe]}} \\
\mykeyword{\textcolor{red}{[chiffres]}.\textcolor{red}{<chiffres>}\textbf{(e|E)}\textcolor{blue}{[exposant]}\textcolor{orange}{[suffixe]}}
\caption{flottant exprimé en décimal}
\end{figure}

\uncover<2->{\mykeyword{\textcolor{red}{chiffres}} et \mykeyword{\textcolor{orange}{exposant}} constituent chacun une suite de chiffres en \myimp{écriture décimale}.}

\uncover<3->{\mykeyword{\textcolor{orange}{suffixe}} détermine \myimp{le type du flottant}.}

\begin{itemize}
    \item<4-> \mykeyword{\textcolor{orange}{f}} ou \mykeyword{\textcolor{orange}{F}} : il s'agit d'un \mykeyword{float} ;
    \item<5-> \textcolor{orange}{aucun suffixe} : il s'agit d'un \mykeyword{double} ;
    \item<6-> \mykeyword{\textcolor{orange}{l}} ou \mykeyword{\textcolor{orange}{L}} : il s'agit d'un \mykeyword{long double}.
\end{itemize}

\end{frame}

\begin{frame}[fragile]

\frametitle{Un petit exemple}

\begin{example}

\begin{lstlisting}[language = c++]
double a = 0.;
double b = 1e3;
float c = 1.2e2f
long double d = .1L
\end{lstlisting}

\end{example}

\end{frame}

\begin{frame}

\frametitle{Les flottants en hexadécimal}

\begin{figure}
\mykeyword{0\textbf{(x|X)}\textcolor{red}{<chiffres>}\textbf{(p|P)}\textcolor{blue}{<exposant>}\textcolor{orange}{[suffixe]}} \\
\mykeyword{0\textbf{(x|X)}.\textcolor{red}{<chiffres>}\textbf{(p|P)}\textcolor{blue}{[exposant]}\textcolor{orange}{[suffixe]}} \\
\mykeyword{0\textbf{(x|X)}\textcolor{red}{[chiffres]}.\textcolor{red}{<chiffres>}\textbf{(p|P)}\textcolor{blue}{[exposant]}\textcolor{orange}{[suffixe]}}
\caption{flottant exprimé en hexadécimal}
\end{figure}

\uncover<2->{\mykeyword{\textcolor{red}{chiffres}} est, cette fois-ci, une séquence de chiffres en \myimp{écriture hexadécimale}.}

\uncover<4->{\mykeyword{\textcolor{blue}{exposant}} est constitué d'une séquence de chiffres en \myimp{écriture décimale}.}

\mywarn<5->{\mykeyword{\textcolor{orange}{exposant}} n'implique plus une multiplication par une puissance de 10 mais par une puissance de 2.}

\end{frame}

\begin{frame}[fragile]

\frametitle{Un petit exemple pour la route}

\begin{example}

\begin{lstlisting}[language = c++]
long double a = 0x10P2l; // a vaut 64
float b       = 0X.1p4F; // b vaut 1
double c      = 0X1.1p4; // c vaut 17
\end{lstlisting}

\end{example}

\end{frame}

\subsubsection{Les littéraux pour les caractères}

\begin{frame}

\frametitle{Les littéraux pour les caractères}

\begin{figure}
\mykeyword{\textcolor{orange}{[préfixe]}'\textcolor{red}{[caractère]}'}
\caption{les littéraux pour les caractères}
\end{figure}

\uncover<2->{\mykeyword{\textcolor{orange}{préfixe}} détermine le type du caractère :}

\begin{itemize}
    \item<3-> \textcolor{orange}{sans préfixe} : il s'agit d'un \mykeyword{char} ;
    \item<4-> \mykeyword{\textcolor{orange}{u8}} : c'est un \mykeyword{char} (cependant, il s'agit d'\myimp{un caractère Unicode} encodé en \myimp{UTF-8} sur \myimp{une seule unité de code}) ;
    \item<5-> \mykeyword{\textcolor{orange}{u}} : il s'agit d'un \mykeyword{char16\_t} ;
    \item<6-> \mykeyword{\textcolor{orange}{U}} : il s'agit d'un \mykeyword{char32\_t} ;
    \item<7-> \mykeyword{\textcolor{orange}{L}} : il s'agit d'un \mykeyword{wchar\_t} ;
\end{itemize}

\end{frame}

\subsection{Les tableaux}

\begin{frame}

\frametitle{Définition d'un tableau}

\begin{definition}
\mydef{Un tableau} est une succession continue d'éléments de même type en mémoire
\end{definition}

\begin{uncoverenv}<2->

\begin{figure}
\mykeyword{\textcolor{red}{<type>} \textcolor{blue}{<identifiant>}[\textcolor{teal}{<taille>}];}
\caption{définition d'un tableau}
\end{figure}

\end{uncoverenv}

\uncover<2->{On crée un tableau de \mykeyword{\textcolor{teal}{taille}} éléments de type \mykeyword{\textcolor{red}{type}} que l'on identifie par le nom \mykeyword{\textcolor{blue}{identifiant}}.}

\mywarn<3->{La taille d'un tableau est statique : elle ne peut pas changer au cours de l'exécution du programme.}

\uncover<4->{La taille d'un tableau caractérise son type.}

\begin{example}<5->
\mykeyword{int[5]} et \mykeyword{int[2]} sont tous deux des tableaux d'entiers mais ce sont deux types distincts.
\end{example}

\end{frame}

\begin{frame}[fragile]

\frametitle{Accès aux éléments}

Si on veut accéder au \mykeyword{\textcolor{teal}{n}}-ième élément de \mykeyword{\textcolor{red}{identifiant}} (les tableaux sont indexés à partir de 0), on utilise la syntaxe suivante :

\begin{figure}
\mykeyword{\textcolor{blue}{<identifiant>}[\textcolor{teal}{<n>}];}
\caption{accès aux éléments d'un tableau}
\end{figure}

\mywarn<2->{Il n'y a aucune vérification de la taille du tableau : on accède à la \mykeyword{\textcolor{teal}{n}}-ième addresse mémoire suivant celle du premier élément du tableau.}

\begin{example}<3->

On a donc :

\begin{lstlisting}[language = c++]
int a[5] = {1, 2, 3, 4, 5};
int a[10]; // cette ligne est valide
\end{lstlisting}

\end{example}

\end{frame}

\begin{frame}[fragile]

\frametitle{Une petite astuce}

Lorsque l'on initialise un tableau avec un nombre d'éléments données, on n'est pas obligé de spécifié sa \mykeyword{\textcolor{teal}{taille}} : elle est déterminée par inférence par le compilateur.

\begin{example}<2->

Les deux lignes suivantes sont valides et équivalentes :

\begin{lstlisting}[language = c++]
int tableau[5] = {1, 2, 3, 4, 5};
int tableau[] = {1, 2, 3, 4, 5}; // tableau est cependant bien de type int[5]
\end{lstlisting}

\end{example}

\end{frame}

\begin{frame}[fragile]

\frametitle{Une question de taille}

La taille du tableau doit être \myimp{déterminée}, \myimp{strictement positive} et \myimp{suffisante}.

\begin{example}<2->

Ainsi, les deux lignes suivantes conduisent à une erreur à la compilation.

\begin{lstlisting}[language = c++]
int tableau[]; // quelle est la taille de tableau ?
int tableau[0]; // certains compilateurs parviennent à la traiter mais ce comportement n'est pas standard
\end{lstlisting}

De plus, \mykeyword{\textcolor{teal}{taille}} ne doit pas nécessairement être strictement égale au nombre d'éléments lors de l'initialisation.

\begin{lstlisting}[language = c++]
int tableau[5] = {1, 2}; // tout va bien
int tableau[1] = {1, 2}; // c'est le drame
\end{lstlisting}

\end{example}

\end{frame}

\subsection{Initialisation de variables}

\begin{frame}

\frametitle{Initialisation de variables}

\mydef{L'initialisation} d'une variable consiste en l'affectation d'une valeur initiale à celle-ci.

\uncover<2->{En C++, il existe \myimp{sept différents types d'initialisation} :}

\begin{itemize}
    \item<3-> \alert<10->{\emph{default initialization}},
    \item<4-> \alert<10->{\emph{value initialization}},
    \item<5-> \alert<10->{\emph{copy initialization}},
    \item<6-> \alert<10->{\emph{list initialization}},
    \item<7->\emph{direct initialization},
    \item<8-> \emph{aggregate initialization},
    \item<9-> \emph{reference initialization}.
\end{itemize}

\uncover<10->{Pour l'instant, on n'abordera que les quatre premières notions.}

\end{frame}

\begin{frame}

\frametitle{default initialization}

La syntaxe utilisée est la suivante :

\begin{figure}
\mykeyword{\textcolor{red}{<type>} \textcolor{blue}{<identifiant>};}
\caption{default initialization}
\end{figure}

Quand on initialise une variable avec cette syntaxe, rien de particulier n'est réalisé hormis l'allocation de la zone mémoire correspondante dans \myimp{la pile d'exécution} : elle a donc une valeur quelconque qui dépend de ce qu'il y avait avant à cette adresse mémoire.

\end{frame}

\begin{frame}[fragile]

\frametitle{Une valeur quelconque ?}

\begin{example}

\begin{lstlisting}[language = c++]
#include <iostream>

using namespace std;

int main()
{
    {
        int tableau[] = {1, 2, 3, 4, 5};
    } // tableau « meurt » ici

    int un_autre_tableau[1];

    cout << un_autre_tableau[2] << '\n';
}
\end{lstlisting}

Dans l'exemple ci-dessus, que va-t-il se passer ?

\myanswer<2->{Il sera imprimé \enquote{3} car c'est la valeur qui a été affectée à cette zone mémoire lors de l'initialisation de \mykeyword{tableau}.}

\end{example}

\end{frame}

\begin{frame}

\frametitle{value initialization}

Les deux syntaxes suivantes sont acceptées :

\begin{figure}
\mykeyword{\textcolor{red}{<type>}();} \\
\mykeyword{\textcolor{red}{<type>} \textcolor{blue}{<identifiant>}\{\};}
\caption{value initialization}
\end{figure}

Quand on initialise une variable avec cette syntaxe, la zone mémoire correspondante à la variable \mykeyword{\textcolor{blue}{identifiant}} est initialisée à 0.

\mywarn<2->{La première syntaxe retourne une \emph{rvalue}.}

\end{frame}

\begin{frame}[fragile]

\frametitle{copy initialization}

La syntaxe acceptée est la suivante :

\begin{figure}
\mykeyword{\textcolor{red}{<type>} \textcolor{blue}{<identifiant>} = \textcolor{teal}{<autre>}};
\caption{copy initialization}
\end{figure}

Le principe de fonctionnement est le même que pour le \emph{direct initialization}.

\end{frame}

\begin{frame}

\frametitle{list initialization}

On pose \( n \in \mathbb{N}^* \).

Les deux syntaxes suivantes sont acceptées :

\begin{figure}
\mykeyword{\textcolor{red}{<type>} \textcolor{blue}{<identifiant>}\{\textcolor{teal}{<argument1>}, \dots, \textcolor{teal}{<argument\( n \)>}\};} \\
\mykeyword{\textcolor{red}{<type>} \textcolor{blue}{<identifiant>} = \{\textcolor{teal}{<argument1>}, \dots, \textcolor{teal}{<argument\( n \)>}\};}
\caption{list initialization}
\end{figure}

Dans le premier cas, on parle de \myimp{direct list initialization}, dans le second, de \myimp{copy list initialization}.

Ces syntaxes ne servent (pour l'instant) que pour l'initialisation de tableaux et elles sont toutes les deux équivalentes.

\end{frame}

\begin{frame}[fragile]

\frametitle{Pour bien comprendre les différents types d'initialisation}

\begin{example}

\begin{lstlisting}[language = c++]
int a;
int d = 8;
int e = {16};
int f = int();
int g{};
int h[] = {1, 2, 3};
\end{lstlisting}

Dans l'exemple ci-dessus, quels sont les types d'initialisation utilisée ?

\begin{enumerate}
    \myanswerize

    \item<visible@2-> \emph{default initialization},
    \item<visible@3-> \emph{copy initialization},
    \item<visible@4-> \emph{copy initialization (\emph{narrowing} interdit)},
    \item<visible@5-> \emph{value initialization}, puis \emph{copy initialization},
    \item<visible@6-> \emph{value initialization}.
    \item<visible@7-> \emph{copy list initialization}.
\end{enumerate}

\end{example}

\end{frame}

\begin{frame}[fragile]

\frametitle{Un autre mot clef pour la route}

En C++, il est possible d'utiliser le mot clef \mykeyword{auto} \myimp{lors de l'initialisation d'une variable}.

\begin{figure}
\mykeyword{auto \textcolor{blue}{<identifiant>} = \textcolor{teal}{<autre>};}
\caption{Utilisation de \mykeyword{auto}}
\end{figure}

Le type de \mykeyword{\textcolor{blue}{identifiant}} est déterminé à partir de \mykeyword{\textcolor{teal}{autre}} par inférence.

\begin{example}<2->

\begin{lstlisting}[language = c++]
auto a = 0.; // a est un double
auto b = 0.f; // b est un float
auto c = 5uLL; // c est un unsigned long long
auto d = false; // d est un bool
auto e; // erreur car il n'est pas possible de déterminer de type
\end{lstlisting}

\end{example}

\end{frame}

\begin{frame}

\centering\Large

Merci pour votre écoute.

\end{frame}

\end{document}
