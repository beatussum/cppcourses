\documentclass{cppcourses}

\title[Formation C++ 17]{
    Formation C++ 17 \no 03 \\
    \small{Pointeurs}
}

\begin{document}

\frame{\titlepage}
\frame{\tableofcontents[pausesections, pausesubsections]}

\section{Avant-propos}

\begin{frame}

\frametitle{Avant-propos}

Avant de commencer, il est important de rappeler que ce cours est réalisé par \myimp{un étudiant}.

Par conséquent, il n'a pas la même fiabilité qu'un cours dispensé par \myimp{un réel enseignant de l'ENSIMAG}.

N'utilisez pas ce cours comme \myimp{un argument d'autorité} !

Si un professeur semble, a posteriori, contredire des éléments apportés par ce cours, \myimp{il a très probablement raison}.

Ce document est \myimp{vivant} : je veillerai à corriger les coquilles ou erreurs plus problématiques.

\end{frame}

\section{Introduction des pointeurs}

\begin{frame}

\frametitle{Qu'est-ce qu'un pointeur ?}

\begin{definition}
\mydef{Un pointeur} qui pointe vers un objet donné représente \myimp{l'adresse mémoire du premier octet dudit objet}.
\end{definition}

\end{frame}

\subsection{Déclaration d'un pointeur}

\subsubsection{Les pointeurs \emph{classiques}}

\begin{frame}

\frametitle{Les pointeurs \emph{classiques}}

\begin{figure}
\mykeyword{\textcolor{orange}{<type>}* \textcolor{red}{<identifiant>};}
\caption{Déclaration d'un pointeur \emph{classiques}}
\end{figure}

Avec la syntaxe ci-dessus, on déclare un pointeur appelé \mykeyword{\textcolor{red}{identifiant}} et pointant vers un objet de type \mykeyword{\textcolor{orange}{type}}.

\begin{remark}<2->
La taille d'un pointeur en mémoire est toujours la même peu importe la valeur de \mykeyword{\textcolor{orange}{type}} : \myimp{elle est égale à un mot}.
\end{remark}

\end{frame}

\subsubsection{Les pointeurs de tableaux}

\begin{frame}

\frametitle{Les pointeurs de tableau}

\begin{remark}
On rappelle que \mykeyword{\textcolor{orange}{<type>}[\textcolor{blue}{<N>}]} caractérise le type d'un tableau. Ainsi, pour \mykeyword{\textcolor{blue}{N}} distinct, deux tableaux sont de type distinct.
\end{remark}

Pour déclarer \myimp{un pointeur de tableau}, il faut donc préciser la taille du tableau référencé. On utilise la syntaxe suivante :

\begin{figure}
\mykeyword{\textcolor{orange}{<type>} (*\textcolor{red}{<identifiant>})[\textcolor{blue}{<N>}];}
\caption{Déclaration d'un pointeur de tableau}
\end{figure}

\end{frame}

\begin{frame}

\frametitle{Une taille inconnue}

\begin{remark}
Le type \mykeyword{\textcolor{orange}{<type>} (*\textcolor{red}{<identifiant>})[]} existe mais il est à éviter.
\end{remark}

\begin{remark}<2->
La règle est qu'il existe \myimp{une conversion implicite} de \mykeyword{\textcolor{orange}{<type>} (*\textcolor{red}{<identifiant>})[\textcolor{blue}{<N>}]} vers \mykeyword{\textcolor{orange}{<type>} (*\textcolor{red}{<identifiant>})[]}, mais, dans l'autre sens, il faut \myimp{une conversion explicite}.
\end{remark}

\end{frame}

\subsubsection{Les pointeurs de fonction}

\begin{frame}

\frametitle{Les pointeurs de fonction}

\begin{definition}
\mydef{Un pointeur de fonction} est un pointeur qui référence une fonction, un morceau du code exécutable, plutôt qu'une donnée.
\end{definition}

\begin{uncoverenv}<2->

Pour \( n \in \mathbb{N} \), la syntaxe utilisé pour en définir un est la suivante :

\begin{figure}
\mykeyword{\textcolor{orange}{<type>} (*\textcolor{red}{<identifiant>})(\textcolor{magenta}{<type de \_\_p1>}, \dots, \textcolor{magenta}{<type de \_\_p\( n \)>});} \\
\caption{Déclaration d'un pointeur de fonction}
\end{figure}

\end{uncoverenv}

\end{frame}

\subsection{Pointeurs nuls}

\begin{frame}

\frametitle{Pointeurs nuls}

\myimp{N'importe quel type de pointeur} peut avoir \myimp{la valeur \mykeyword{nullptr}} qui signifie que celui-ci pointe vers l'adresse nulle, une adresse invalide : il pointe donc vers \textit{rien}.

\begin{remark}<2->

Un pointeur peut être implicitement converti en booléen :

\begin{enumerate}
    \item<3-> \myimp{s'il est nul}, alors il est \myimp{faux} ;
    \item<4-> \myimp{sinon}, il est \myimp{vrai}.
\end{enumerate}

\end{remark}

\begin{remark}<5->
\mykeyword{nullptr} a pour type \mykeyword{std::nullptr\_t}.
\end{remark}

\end{frame}

\begin{frame}[fragile]

\frametitle{segmentation fault}

\begin{example}

\begin{lstlisting}[language=c++]
#include <iostream>

int main()
{
    int* p_a = nullptr;

    std::cout << *p_a << '\n';
}
\end{lstlisting}

Dans l'exemple ci-dessus, que va-t-il se passer ?

\myanswer<2->{On va avoir une erreur de segmentation car \mykeyword{p\_a} a une valeur invalide.}

\end{example}

\begin{definition}<visible@3->
\mydef{Une erreur de segmentation} (en anglais \mydef{\enquote{segmentation fault}}) est un plantage d'une application qui a tenté d'accéder à un emplacement mémoire qui ne lui était pas alloué.
\end{definition}

\end{frame}

\subsection{Pointeurs et tableaux}

\begin{frame}[fragile]

\frametitle{Pointeurs et tableaux}

On peut confondre \myimp{un tableau} avec \myimp{le pointeur pointant sur son premier élément}.

\begin{remark}<2->
Il existe donc une conversion implicite entre \mykeyword{\textcolor{orange}{<type>}[\textcolor{blue}{<N>}]} et \mykeyword{\textcolor{orange}{<type>}*}.
\end{remark}

\begin{warning}<3->
Il ne faut pas confondre cela avec la notion de \myimp{pointeur de tableau}.
\end{warning}

\begin{example}<4->

\begin{lstlisting}[language=c++]
int array[5];
int* p_first = array;
int (*p_array)[5] = &array;
\end{lstlisting}

\end{example}

\end{frame}

\subsection{Manipulations élémentaires des pointeurs}

\subsubsection{Opérations élémentaires et déréférencement}

\begin{frame}

\frametitle{Manipulations élémentaires des pointeurs}

Il est bon de visualiser ce qu'est un pointeur en mémoire comme étant un entier (\myimp{l'adresse de l'objet pointé}).

\begin{uncoverenv}<2->

Ainsi, certaines opérations sont possibles avec \mykeyword{a} et \mykeyword{b} \myimp{deux pointeurs de même types} et \mykeyword{n} \myimp{un intégral}.

\begin{columns}
    \begin{column}{0.5\textwidth}
        \begin{figure}
\mykeyword{a \textcolor{red}{+} n} \\
\mykeyword{a \textcolor{red}{-} n}
\caption{Opérateurs arithmétiques}
        \end{figure}
    \end{column}
    \begin{column}{0.5\textwidth}
        \begin{figure}
\mykeyword{a \textcolor{red}{+=} n} \\
\mykeyword{a \textcolor{red}{-=} n}
\caption{Opérateurs d'affectation associés}
        \end{figure}
    \end{column}
\end{columns}

\end{uncoverenv}

\begin{uncoverenv}<3->

\begin{figure}
\mykeyword{a \textcolor{red}{==} b} \\
\mykeyword{a \textcolor{red}{!=} b} \\
\mykeyword{a \textcolor{red}{<} b} \\
\mykeyword{a \textcolor{red}{>} b} \\
\mykeyword{a \textcolor{red}{<=} b} \\
\mykeyword{a \textcolor{red}{>=} b}
\caption{Opérateurs de comparaison}
\end{figure}

\end{uncoverenv}

\end{frame}

\begin{frame}

\frametitle{Une remarque importante}

Pour calculer la distance séparant deux adresses mémoires de \myimp{deux pointeurs de même type} \mykeyword{a} et \mykeyword{b}, on utilise la syntaxe suivante :

\begin{figure}
\mykeyword{b \textcolor{red}{-} a}
\caption{Soustraction de pointeurs}
\end{figure}

\begin{warning}<2->
Le résultat de cette opération n'est pas du tout le même si \mykeyword{a} désigne \myimp{un intégral}.
\end{warning}

\end{frame}

\begin{frame}

\frametitle{De nouveaux opérateurs}

D'autres opérations sont possibles avec les pointeurs. \mykeyword{a} désigne \myimp{un pointeur} et \mykeyword{u} \myimp{une variable quelconque}.

\begin{columns}[c]
    \begin{column}{0.5\textwidth}<2->

\begin{figure}
\mykeyword{\textcolor{red}{*}a}
\caption{Opérateur de déréférencement}
\end{figure}

\uncover<3->{On \myimp{déréférence le pointeur} : en d'autres termes, on accède à la valeur pointé par le pointeur.}

\uncover<4->{Si \mykeyword{a} est de type \mykeyword{\textcolor{orange}{type}*}, le type de \mykeyword{\textcolor{red}{*}a} est \mykeyword{\textcolor{orange}{type}\&} (\myimp{une référence} vers \mykeyword{a}).}

    \end{column}

    \begin{column}{0.5\textwidth}<5->

\begin{figure}
\mykeyword{\textcolor{red}{\&}u}
\caption{Opérateur d'\enquote{adresse de}}
\end{figure}

\uncover<6->{On récupère l'adresse mémoire de \mykeyword{u}.}

\uncover<7->{Si \mykeyword{u} est de type \mykeyword{\textcolor{orange}{type}}, le type de \mykeyword{\textcolor{red}{\&}u} est \mykeyword{\textcolor{orange}{type}*}.}

    \end{column}
\end{columns}

\end{frame}

\begin{frame}[fragile]

\frametitle{Une mise en application}

\begin{example}

\begin{lstlisting}[language=c++]
#include <iostream>
#include <cstdint>

int main()
{
    std::uint8_t a = 10;
    std::uint16_t b = 257;
    std::uint8_t* p_a = &a;

    std::cout << static_cast<int>(*p_a) << '\n';
    std::cout << static_cast<int>(*(p_a + 1)) << '\n';
}
\end{lstlisting}

Dans l'exemple ci-dessus, que va-t-il se passer ?

\myanswer<2->{Il sera imprimé en console \enquote{10}, puis \enquote{1} (resp. \enquote{255}) sur une machine petit-boutiste (resp. gros-boutiste).}

\end{example}

\end{frame}

\subsubsection{Un opérateur bien pratique}

\begin{frame}[fragile]

\frametitle{Un opérateur bien pratique}

\begin{example}

\begin{lstlisting}[language = c++]
int* a = {1, 2, 3, 4, 5}; // un pointeur vers le premier élément du tableau

// Si on veut accéder au troisème élément du tableau, il faut faire :

std::cout << *(a + 2) << '\n';
\end{lstlisting}

\begin{uncoverenv}<2->

Comme cette opération n'est pas pratique et très courante, on introduit un nouvel opérateur qui

\begin{enumerate}
    \item incrémente,
    \item puis déréférence le pointeur.
\end{enumerate}

\end{uncoverenv}

\begin{uncoverenv}<3->

\begin{lstlisting}[language = c++]
int* a = {1, 2, 3, 4, 5}; // un pointeur vers le premier élément du tableau

// Maintenant, on peut faire :

std::cout << a[2] << '\n';
\end{lstlisting}

\end{uncoverenv}

\end{example}

\end{frame}

\begin{frame}

\frametitle{Détails syntaxiques}

\mykeyword{a} désigne {un pointeur} et \mykeyword{n} \myimp{un intégral}.

\begin{figure}
\mykeyword{a\textcolor{red}{[}n\textcolor{red}{]}}
\caption{Opérateurs d'indice}
\end{figure}

\uncover<2->{Si \mykeyword{a} est de type \mykeyword{\textcolor{orange}{type}}, alors \mykeyword{a\textcolor{red}{[}n\textcolor{red}{]}} a pour type \mykeyword{\textcolor{orange}{type}\&}.}

\end{frame}

\subsection{Pointeurs multiples}

\begin{frame}

\frametitle{Un petit rappel}

\begin{remark}
\myimp{En C et C++}, une chaîne de caractère est, \myimp{par convention}, \myimp{un tableau de caractères} dont la fin est déterminée par \myimp{le caractère nulle} (\myimp{\mykeyword{'\textbackslash 0'}}).
\end{remark}

\begin{remark}<2->
On appelle le caractère \enquote{\myimp{\textbackslash}} \myimp{le caractère d'échappement}.
\end{remark}

\begin{warning}<3->
Il ne faut pas confondre \myimp{le caractère nulle} (\myimp{\mykeyword{'\textbackslash 0'}}) et \myimp{le caractère zéro} (\myimp{\mykeyword{'0'}}).
\end{warning}

\end{frame}

\begin{frame}[fragile]

\frametitle{Un petit exemple}

\begin{example}

\begin{lstlisting}[language = c++]
char a[] = "Hello!";
char b[] = {'H', 'e', 'l', 'l', 'o', '!', '\0'};
\end{lstlisting}

Dans l'exemple ci-dessus, \mykeyword{a} et \mykeyword{b} ont, en fait, la même valeur.

\end{example}

\begin{warning}<2->
Il ne faut donc pas oublier la présence du caractère nul quand on veut déterminer la taille du tableau.
\end{warning}

\end{frame}

\begin{frame}

\frametitle{Pointeurs multiples}

On reprend la syntaxe suivante :

\begin{figure}
\mykeyword{\textcolor{orange}{<type>}* \textcolor{red}{<identifiant>};}
\caption{Déclaration d'un pointeur \emph{classiques}}
\end{figure}

On remarque que \mykeyword{\textcolor{orange}{<type>}*} définie également un type : un pointeur est un type valide.

Ainsi, il existe des pointeurs de pointeur.

\end{frame}

\begin{frame}[fragile]

\frametitle{Un cas concret pour bien comprendre}

\begin{example}

\begin{lstlisting}[language=c++]
#include <iostream>

int main()
{
    const char* sentence[] = {"Hello", "world!"};

    for (const char** i = sentence; i != (sentence + 2); ++i) {
        std::cout << *i << ' ';
    }

    std::cout << '\n';
}
\end{lstlisting}

Que fait le programme ci-dessus ?

\myanswer<2->{Il imprime sur la console \enquote{Hello world!}.}

\end{example}

\end{frame}

\begin{frame}[fragile]

\frametitle{Le mot-clef \mykeyword{const}}

Le mot-clef \mykeyword{const} spécifie qu'une variable est \myimp{immutable}.

\begin{remark}<2->
\mykeyword{const} caractérise le type.
\end{remark}

\begin{example}<3->

\mykeyword{char} et \mykeyword{const char} sont deux types distincts.

Cependant, il existe une conversion implicite de \mykeyword{\textcolor{orange}{<type>}} vers \mykeyword{const \textcolor{orange}{<type>}}.

\end{example}

\begin{example}<4->

\begin{lstlisting}[language=c++]
int a = 5;
const int b = a;
++b; // il y a erreur
b = 7; // là aussi
\end{lstlisting}

\end{example}

\end{frame}

\begin{frame}

\frametitle{Constants pointeurs ou pointeurs constants}

\mykeyword{const \textcolor{orange}{<type>}*} et \mykeyword{\textcolor{orange}{<type>} *const} ne signifie pas la même chose.

\begin{enumerate}
    \item<2-> \myimp{Pour le premier}, le pointeur pointe vers une entité de type \mykeyword{const \textcolor{orange}{<type>}} ;
    \item<3-> \myimp{pour le second}, le pointeur est lui-même immutable.
\end{enumerate}

\uncover<4->{La règle pour \mykeyword{const} est la suivante :}

\begin{enumerate}
    \item<5-> il s'applique sur \myimp{le type de gauche},
    \item<6-> \myimp{s'il n'y a pas de type à gauche}, alors il s'applique sur \myimp{le type de droite}.
\end{enumerate}

\begin{remark}<7->
\mykeyword{const \textcolor{orange}{<type>}} et \mykeyword{\textcolor{orange}{<type>} const} désignent donc les mêmes types.
\end{remark}

\end{frame}

\begin{frame}

\centering\Large

Merci pour votre écoute.

\end{frame}

\end{document}
